%
% This work is licensed under a Creative Commons Attribution-ShareAlike 4.0 International License.
% http://creativecommons.org/licenses/by-sa/4.0/
%

% DO NOT COMPILE THIS FILE DIRECTLY!
% This is included by the other .tex files.


\begin{frame}
    \includegraphics[scale=0.3]{images/logo-circl-Forensics.png}
    \begin{itemize}
        \item[]
        \item[]
        \item[] 2. Information
    \end{itemize}
\end{frame}


\begin{frame}[fragile]
  \frametitle{2.1 Data in a binary system}
    \begin{itemize}
        \item BIT $\to$ Binary digit
        \item Data stored in binary form
            \begin{itemize}
                \item[] \begin{verbatim}x Bits --> 01010000011010010110111001100111 --> y Bits\end{verbatim}
                \item[] Bit x + 2 = 1
                \item[] Bit x + 3 = 0
                \item[] $\to$ What information is stored within this data?
            \end{itemize}
    \item[]
    \item \emph{"..... information is data arranged in a meaningful way for some perceived purpose ....."} $\to$ Interpretative rules
    \item[]
        \item Grouping, addressing and interpreting
            \begin{itemize}
                \item[] \begin{verbatim}-->   01010000   01101001   01101110   01100111   --> \end{verbatim}
                \item[] \begin{verbatim}      --------   --------   --------   -------- \end{verbatim}
                \item[] \begin{verbatim}-->   Byte 117   Byte 118   Byte 119   Byte 120   --> \end{verbatim}
            \end{itemize}
    \item[]
    \end{itemize}
\end{frame}


\begin{frame}[fragile]
  \frametitle{2.1 Data in a binary system}
    \begin{itemize}
        \item Grouping examples:
        \begin{itemize}
	    \item Nibble:\begin{footnotesize} \texttt{0101 0000 0110 1001 0110 1110 0110 0111} \end{footnotesize} 
            \item Byte:\begin{footnotesize} \texttt{01010000 01101001 01101110 01100111} \end{footnotesize}
            \item Word:\begin{footnotesize} \texttt{0101000001101001 0110111001100111} \end{footnotesize}
            \item Double Word:\begin{footnotesize} \texttt{01010000011010010110111001100111} \end{footnotesize}
	    \item[]
        \end{itemize}
        \item Interpreting:
        \begin{itemize}
	    \item Integer: (Signed, Unsigned)
	    \item Endian: (Big, Little)
            \item Floating Point
            \item Binary Coded Decimal, Packed BCD
	    \item Encoding: (ASCII, ISO8859, Unicode 16L, 16B, 32L, 32B)
	    \item Binary: (ELF, MZ, PE, GIF, JPEG, ZIP, PDF, OLE, ...)
            \item ...
        \end{itemize}
    \end{itemize}
\end{frame}


\begin{frame}[fragile]
  \frametitle{2.2 Number Systems}
    \begin{itemize}
        \item Decimal:
  \begin{lstlisting}[basicstyle=\tiny]
  2145
  ||||_   5 * 10^0 =      5
  |||__   4 * 10^1 =     40
  ||___   1 * 10^2 =    100
  |____   2 * 10^3 =  2,000
		      -----
                      2,145
  \end{lstlisting}
        \item Hexadecimal:
  \begin{lstlisting}[basicstyle=\tiny]
  2A9F
  ||||_  15 * 16^0 =     15
  |||__  09 * 16^1 =    144
  ||___  10 * 16^2 =  2,560
  |____  02 * 16^3 =  8,192
                     ------
                     10,911
  \end{lstlisting}
        \item Binary:
  \begin{lstlisting}[basicstyle=\tiny]
  1111
  ||||_   1 *  2^0 =      1
  |||__   1 *  2^1 =      2
  ||___   1 *  2^2 =      4
  |____   1 *  2^3 =      8
                       = 15
  \end{lstlisting}
    \end{itemize}
\end{frame}


\begin{frame}[fragile]
  \frametitle{2.3 Interpreting binary data: Integer}
  \begin{verbatim}
  0 1 0 1 0 0 0 0
  ---------------
  | | | | | | | |_  0 * 2^0 =   0
  | | | | | | |___  0 * 2^1 =   0
  | | | | | |_____  0 * 2^2 =   0
  | | | | |_______  0 * 2^3 =   0
  | | | |_________  1 * 2^4 =  16
  | | |___________  0 * 2^5 =   0
  | |_____________  1 * 2^6 =  64
  |_______________  0 * 2^7 =   0
                              ---
                               80
  \end{verbatim}
\end{frame}


\begin{frame}[fragile]
  \frametitle{2.3 Interpreting binary data: Signed Integer}
  \begin{verbatim}
  1 0 1 1 1 1 1 1
  ---------------           Two's complement:

  0 1 0 0 0 0 0 0           1. Invert all single bits
  0 1 0 0 0 0 0 1           2. Add the value 1

    |           |

   64           1
   --------------
              -65
  \end{verbatim}
\end{frame}


\begin{frame}[fragile]
  \frametitle{2.4 Exercise: Signed Integer Bytes}
  \begin{verbatim}
  1 1 0 1 1 1 0 0
  ---------------           Two's complement:

                            1. Invert all single bits
                            2. Add the value 1
    
    | | | | | | |

    ? ? ? ? ? ? ?
   --------------
              -
  \end{verbatim}
\end{frame}


\begin{frame}[fragile]
  \frametitle{2.4 Exercise: Signed Integer Bytes}
  \begin{verbatim}
  1 1 0 1 1 1 0 0
  ---------------           Two's complement:

  0 0 1 0 0 0 1 1           1. Invert all single bits
  0 0 1 0 0 1 0 0           2. Add the value 1
    
      |     |

     32     4
   --------------
              -36
  \end{verbatim}
\end{frame}


\begin{frame}[fragile]
  \frametitle{2.5 From Bin to Hex}
    \begin{itemize}
        \item[]
        \item[] Example:
            \begin{itemize}
                \item[] \begin{verbatim} 0001 1000    0101 0101    0000 1111    1010 0110\end{verbatim}
                \item[] \begin{verbatim} ---------    ---------    ---------    ---------\end{verbatim}
		\item[] \begin{verbatim}    1 8          5 5          0 F          A 6   \end{verbatim}
                \item[] 
            \end{itemize}
        \item[]
        \item[] Exercise:
            \begin{itemize}
                \item[] \begin{verbatim} 1001 0110    1010 0101    0000 1111    1100 0011\end{verbatim}
                \item[] \begin{verbatim} ---------    ---------    ---------    ---------\end{verbatim}
                \item[] \begin{verbatim}                                                 \end{verbatim}
                \item[] 
            \end{itemize}
    \end{itemize}
\end{frame}


\begin{frame}[fragile]
  \frametitle{2.5 From Bin to Hex}
    \begin{itemize}
        \item[]
        \item[] Exercise:
            \begin{itemize}
                \item[] \begin{verbatim} 1001 0110    1010 0101    0000 1111    1100 0011\end{verbatim}
                \item[] \begin{verbatim} ---------    ---------    ---------    ---------\end{verbatim}
                \item[] \begin{verbatim}                                                 \end{verbatim}
                \item[] 
            \end{itemize}
        \item[]
        \item[] Results:
            \begin{itemize}
                \item[] \begin{verbatim} 1001 0110    1010 0101    0000 1111    1100 0011\end{verbatim}
                \item[] \begin{verbatim} ---------    ---------    ---------    ---------\end{verbatim}
                \item[] \begin{verbatim}    9 6          A 5          0 F          C 3   \end{verbatim}
                \item[] 
            \end{itemize}
    \end{itemize}
\end{frame}


\begin{frame}[fragile]
  \frametitle{2.6 Big Endian and Little Endian}
	\begin{lstlisting}
Large values (Example 256): Big Endian representation:

    2^: 15 14 13 12 11 10  9  8    7  6  5  4  3  2  1  0
         -  -  -  -  -  -  -  -    -  -  -  -  -  -  -  - 
         0  0  0  0  0  0  0  1    0  0  0  0  0  0  0  0 
         -  -  -  -  -  -  -  -    -  -  -  -  -  -  -  - 
Address: 10.000                    10.001 
           
            
Large values (Example 256): Little Endian representation:

     2^: 7  6  5  4  3  2  1  0   15 14 13 12 11 10  9  8 
         -  -  -  -  -  -  -  -    -  -  -  -  -  -  -  - 
         0  0  0  0  0  0  0  0    0  0  0  0  0  0  0  1 
         -  -  -  -  -  -  -  -    -  -  -  -  -  -  -  - 
Address: 10.000                    10.001 
	\end{lstlisting}
\end{frame}


\begin{frame}[fragile]
  \frametitle{2.6 Exercise: Little Endian}
    \begin{itemize}
        \item[] Read and interpret this little endian 'word'
            \begin{itemize}
                \item[] \begin{verbatim}        ------------- \end{verbatim}
                \item[] \begin{verbatim} 0x96A5 | 9 6 | A 5 | \end{verbatim}
                \item[] \begin{verbatim}        ------------- \end{verbatim}
                \item[] \begin{verbatim}                      \end{verbatim}
                \item[] \begin{verbatim}                      \end{verbatim}
                \item[] \begin{verbatim}                      \end{verbatim}
                \item[] \begin{verbatim}                      \end{verbatim}
                \item[] \begin{verbatim}                      \end{verbatim}
                \item[] \begin{verbatim}        ------------- \end{verbatim}
                \item[] \begin{verbatim} 0x     |     |     | =  \end{verbatim}
                \item[] \begin{verbatim}        ------------- \end{verbatim}
            \end{itemize}
    \end{itemize}
\end{frame}


\begin{frame}[fragile]
  \frametitle{2.6 Exercise: Little Endian}
    \begin{itemize}
        \item[] Read and interpret this little endian 'word'
            \begin{itemize}
                \item[] \begin{verbatim}        ------------- \end{verbatim}
                \item[] \begin{verbatim} 0x96A5 | 9 6 | A 5 | \end{verbatim}
                \item[] \begin{verbatim}        ------------- \end{verbatim}
                \item[] \begin{verbatim}          ---   ---   \end{verbatim}
                \item[] \begin{verbatim}             \ /      \end{verbatim}
                \item[] \begin{verbatim}              X       \end{verbatim}
                \item[] \begin{verbatim}             / \      \end{verbatim}
                \item[] \begin{verbatim}          ---   ---   \end{verbatim}
                \item[] \begin{verbatim}        ------------- \end{verbatim}
                \item[] \begin{verbatim} 0xA596 | A 5 | 9 6 | = 42,390 \end{verbatim}
                \item[] \begin{verbatim}        ------------- \end{verbatim}
            \end{itemize}
    \end{itemize}
\end{frame}


\begin{frame}[fragile]
  \frametitle{2.6 Exercise: Little Endian}
    \begin{itemize}
        \item[] Read and interpret this little endian 'double word'
            \begin{itemize}
                \item[] \begin{verbatim}            ------------------------- \end{verbatim}
                \item[] \begin{verbatim} 0x1B2A0100 | 1 B | 2 A | 0 1 | 0 0 | \end{verbatim}
                \item[] \begin{verbatim}            ------------------------- \end{verbatim}
                \item[] \begin{verbatim}                                      \end{verbatim}
                \item[] \begin{verbatim}                                      \end{verbatim}
                \item[] \begin{verbatim}                                      \end{verbatim}
                \item[] \begin{verbatim}                                      \end{verbatim}
                \item[] \begin{verbatim}                                      \end{verbatim}
                \item[] \begin{verbatim}            ------------------------- \end{verbatim}
                \item[] \begin{verbatim} 0x         |     |     |     |     | =  \end{verbatim}
                \item[] \begin{verbatim}            ------------------------- \end{verbatim}
            \end{itemize}
    \end{itemize}
\end{frame}


\begin{frame}[fragile]
  \frametitle{2.6 Exercise: Little Endian}
    \begin{itemize}
        \item[] Read and interpret this little endian 'double word'
            \begin{itemize}
                \item[] \begin{verbatim}            ------------------------- \end{verbatim}
                \item[] \begin{verbatim} 0x1B2A0100 | 1 B | 2 A | 0 1 | 0 0 | \end{verbatim}
                \item[] \begin{verbatim}            ------------------------- \end{verbatim}
                \item[] \begin{verbatim}              ---   ---   ---   ---   \end{verbatim}
                \item[] \begin{verbatim}                 \___  \ /  ___/      \end{verbatim}
                \item[] \begin{verbatim}                  ___   X   ___       \end{verbatim}
                \item[] \begin{verbatim}                 /     / \     \      \end{verbatim}
                \item[] \begin{verbatim}              ---   ---   ---   ---   \end{verbatim}
                \item[] \begin{verbatim}            ------------------------- \end{verbatim}
                \item[] \begin{verbatim} 0x00012A1B | 0 0 | 0 1 | 2 A | 1 B | = 76,315 \end{verbatim}
                \item[] \begin{verbatim}            ------------------------- \end{verbatim}
            \end{itemize}
    \end{itemize}
\end{frame}


\begin{frame}[fragile]
  \frametitle{2.7 Example: Others}
  \begin{itemize}
    \item[] BCD / PBCD
  \begin{verbatim}
     2        9        1    
  -------- -------- --------
  00000010 00001001 00000001     0110 1111 0000 1001
                                 ---- ---- ---- ----
                                  6    na   0    9
  \end{verbatim}
    \item[] ASCII
  \begin{verbatim}
  01110000  01101001  01101110  01100111
  --------  --------  --------  --------
    0x70      0x65      0x6E      0x67
     112       105       110       103
      p         i         n         g
  \end{verbatim}
  \end{itemize}
\end{frame}


\begin{frame}[fragile]
  \frametitle{2.8 Data structures: Example}
\begin{lstlisting}[basicstyle=\tiny]
  0                                           15
 -------------------------------------------------
 |4B|50|08|0E|00|74|65|73|74|2E|74|78|74|22|48|65|
 -------------------------------------------------
 

 16                                           31
 -------------------------------------------------
 |6C|6C|6F|20|57|6F|72|6C|64|22|0D|4B|50|1A|11|01|
 -------------------------------------------------
  

 32                                           47
 -------------------------------------------------
 |..|..|..|..|..|..|..|..|..|..|..|..|..|..|..|..|
 -------------------------------------------------
 .
 .
 .
 .
 .
 .
 .
 .
 .
\end{lstlisting}
\end{frame}


\begin{frame}[fragile]
  \frametitle{2.8 Data structures: Example}
\begin{lstlisting}[basicstyle=\tiny]
  0                                           15
 -------------------------------------------------
 |4B|50|08|0E|00|74|65|73|74|2E|74|78|74|22|48|65|
 -------------------------------------------------
 | K P | 8|  14 | 

 16                                           31
 -------------------------------------------------
 |6C|6C|6F|20|57|6F|72|6C|64|22|0D|4B|50|1A|11|01|
 -------------------------------------------------
  

 32                                           47
 -------------------------------------------------
 |..|..|..|..|..|..|..|..|..|..|..|..|..|..|..|..|
 -------------------------------------------------
 
 
 
Offset Size Description
     0    2 Header signature (ASCII)
     2    1 Lenght of file name (Integer)
     3    2 Lenght of data (Integer little endian)
     5   -- Variable file name (ASCII)
    5+   -- Data (Binary)
\end{lstlisting}
\end{frame}


\begin{frame}[fragile]
  \frametitle{2.8 Data structures: Example}
\begin{lstlisting}[basicstyle=\tiny]
  0                                           15
 -------------------------------------------------
 |4B|50|08|0E|00|74|65|73|74|2E|74|78|74|22|48|65|
 -------------------------------------------------
 | K P | 8|  14 | t  e  s  t  .  t  x t | "  H  e

 16                                           31
 -------------------------------------------------
 |6C|6C|6F|20|57|6F|72|6C|64|22|0D|4B|50|1A|11|01|
 -------------------------------------------------
   l  l  o     W  o  r  l  d  "   | 

 32                                           47
 -------------------------------------------------
 |..|..|..|..|..|..|..|..|..|..|..|..|..|..|..|..|
 -------------------------------------------------
 
 
 
Offset Size Description
     0    2 Header signature (ASCII)
     2    1 Lenght of file name (Integer)
     3    2 Lenght of data (Integer little endian)
     5   -- Variable file name (ASCII)
    5+   -- Data (Binary)
\end{lstlisting}
\end{frame}


\begin{frame}[fragile]
  \frametitle{2.8 Data structures: Example}
\begin{lstlisting}[basicstyle=\tiny]
  0                                           15
 -------------------------------------------------
 |4B|50|08|0E|00|74|65|73|74|2E|74|78|74|22|48|65|
 -------------------------------------------------
 | K P | 8|  14 | t  e  s  t  .  t  x t | "  H  e

 16                                           31
 -------------------------------------------------
 |6C|6C|6F|20|57|6F|72|6C|64|22|0D|4B|50|1A|11|01|
 -------------------------------------------------
   l  l  o     W  o  r  l  d  "   | K P |26| 273 |

 32                                           47
 -------------------------------------------------
 |..|..|..|..|..|..|..|..|..|..|..|..|..|..|..|..|
 -------------------------------------------------
 
 
 
Offset Size Description
     0    2 Header signature (ASCII)
     2    1 Lenght of file name (Integer)
     3    2 Lenght of data (Integer little endian)
     5   -- Variable file name (ASCII)
    5+   -- Data (Binary)
\end{lstlisting}
\end{frame}



\begin{frame}
  \frametitle{2.9 Data, files, context}
    \begin{itemize}
        \item Sequence of Bits + Addressing + Interpretation $\to$ Information
        \item Information $\to$ Stored in files
        \item Where did you find the string "ping"?
            \begin{itemize}
                \item Binary inside TEMP folder
                \item Autorun folder
                \item Registry
                \item Browser history
                \item Command line history
            \end{itemize}
        \item [] $\to$ Data $\to$ Information $\to$ Knowledge
	\item[] 
	\item Files contains data
	\item Files $\to$ Meta data describe files
        \item Files $\to$ File systems organize files and meta data
    \end{itemize}
\end{frame}









