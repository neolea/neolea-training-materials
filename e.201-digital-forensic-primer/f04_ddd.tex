% DO NOT COMPILE THIS FILE DIRECTLY!
% This is included by the other .tex files.


\begin{frame}
    \includegraphics[scale=0.3]{images/logo-circl-Forensics.png}
    \begin{itemize}
        \item[]
        \item[]
        \item[] 4. Disk Cloning / Disk Imaging
    \end{itemize}
\end{frame}


\begin{frame}[fragile]
  \frametitle{4.1 Disk cloning - imaging}
    \begin{itemize}
        \item Clone disk-2-disk
        \begin{itemize}
            \item Different sizes
            \item Wipe target disk!
        \end{itemize}
        \item Clone disk-2-image
        \begin{itemize}
            \item Clear boundaries
            \item One big file
            \item Break file into chunks
        \end{itemize}
        \item Image file format
        \begin{itemize}
            \item RAW
            \item AFF (Advanced Forensic Format)
            \item EWF (Expert Witness Format)
            \item Please no 3rd party formats
        \end{itemize}
        \item Write-Blockers
        \begin{itemize}
            \item Hardware
        \end{itemize}
    \end{itemize}
\end{frame}


\begin{frame}[fragile]
  \frametitle{4.2 Connecting devices}
    \begin{itemize}
        \item \texttt{udev} 
\begin{lstlisting}[basicstyle=\tiny]
udevadm info /dev/sda                 # userspace /dev
udevadm monitor
\end{lstlisting}
        \item \texttt{/dev/}
\begin{lstlisting}[basicstyle=\tiny]
/dev/sd*               # SCSI, SATA
/dev/hd*               # IDE. EIDE
/dev/md*               # RAID
/dev/nvme*n*           # NVME devices

/dev/sda1              # Partition 1 on disk 1
/dev/sda2              # Partition 2 on disk 1
...
\end{lstlisting}
        \item Block Devices
        \begin{itemize}
            \item Attaching
            \item Mounting
        \end{itemize}
    \end{itemize}
\end{frame}


\begin{frame}[fragile]
  \frametitle{4.2 Read partition table}
    \begin{itemize}
        \item \texttt{dmesg} 
\begin{lstlisting}[basicstyle=\tiny]
[106834.127269] sd 6:0:0:0: Attached scsi generic sg1 type 0
[106834.127503] sd 6:0:0:0: [sdb] 15826944 512-byte logical blocks: (8.10 GB/7.54 GiB)
[106834.130380] sd 6:0:0:0: [sdb] Write Protect is off
\end{lstlisting}
        \item \texttt{fdisk -l circl-dfir.dd}
\begin{lstlisting}[basicstyle=\tiny]
Disk circl-dfir.dd: 1536 MB, 1536000000 bytes
4 heads, 7 sectors/track, 107142 cylinders, total 3000000 sectors
Units = sectors of 1 * 512 = 512 bytes
Sector size (logical/physical): 512 bytes / 512 bytes
I/O size (minimum/optimal): 512 bytes / 512 bytes
Disk identifier: 0x8f7e6594

        Device Boot      Start         End      Blocks   Id  System
circl-dfir.dd1            2048     3000000     1498976+   7  HPFS/NTFS/exFAT
\end{lstlisting}
        \item Exercise: Analyze output. Why \texttt{1498976}? $\to$ Conclusions?
    \end{itemize}
\begin{lstlisting}[basicstyle=\tiny]
End:     echo $(( 3000000 * 512 ))                  --> 1536 MB, 1536000000 bytes 
         echo $(( 3000000 * 512 / 1024 / 1024 ))    --> 1464

1498976: echo $(( 1498976 * 2 ))                    --> 2997952
\end{lstlisting}
\end{frame}


\begin{frame}[fragile]
  \frametitle{4.2 Mounting}
    \begin{itemize}
        \item \texttt{mount} 
\begin{lstlisting}[basicstyle=\tiny]
mkdir /mnt/ntfs                             # Create mount point
mount /dev/sdb1 /mnt/ntfs                   # Mounting

mount -o ro,remount /dev/sdb1 /mnt/ntfs     # Re-mounting

umount /mnt/ntfs                            # Un-mounting
umount /dev/sdb1                            # Also un-mounting


# Mounting readonly, no journaling, no executable
mount -o ro,noload,noexec /dev/sdb1 /mnt/ntfs
mount -o ro,noload,noexec,remount /dev/sdb1 /mnt/ntfs


# Mounting with offset. mounting from image files
mount -o ro,noload,noexec,offset=$((512*2048)) circl-dfir.dd /mnt/ntfs


# Mounting NTFS file systems
mount -o ro,noload,noexec,offset=$((512*2048)),
      show_sys_files,streams_interface=windows circl-dfir.dd /mnt/ntfs

\end{lstlisting}
    \end{itemize}
\end{frame}


\begin{frame}[fragile]
  \frametitle{4.3 dd - disk imaging rudimentary}
    \begin{itemize}
        \item[] Copy files from: \texttt{/mnt/ntfs/dd/}
\begin{lstlisting}[basicstyle=\tiny]
$ dd if=img_1.txt of=out_1.txt bs=512

     <input file>      <output file> <block size>
                                      (default)
3+0 records in
3+0 records out
1536 bytes (1.5 kB) copied, 0.000126 s, 12.2 MB/s

$ ll
-rw-rw-r-- 1 hamm hamm 1536 May 16 11:20 img_1.txt
-rw-rw-r-- 1 hamm hamm 1536 May 16 11:16 out_1.txt



$ dd if=img_2.txt of=out_2.txt bs=512
3+1 records in
3+1 records out
1591 bytes (1.6 kB) copied, 0.00016048 s, 9.9 MB/s

$ ll
-rw-rw-r-- 1 hamm hamm 1591 May 16 11:20 img_2.txt
-rw-rw-r-- 1 hamm hamm 1591 May 16 11:26 out_2.txt
\end{lstlisting}
    \end{itemize}
\end{frame}


\begin{frame}[fragile]
  \frametitle{4.3 dd - disk imaging rudimentary}
    \begin{itemize}
        \item[] Demo: \texttt{skip} and \texttt{count} options
\begin{lstlisting}[basicstyle=\tiny]
dd if=img_3.txt bs=512 skip=0 count=1 status=none | less
dd if=img_3.txt bs=512 skip=1 count=1 status=none | less
dd if=img_3.txt bs=512 skip=2 count=1 status=none | less
\end{lstlisting}
        \item[] Exercise: Find the secret password behind sector 3
        \item[] Exercise: Play with \texttt{bs, skip} and \texttt{count} options
\begin{lstlisting}[basicstyle=\tiny]
dd if=img_3.txt bs=1 skip=$((512*3)) count=16 status=none
dd if=img_3.txt bs=16 skip=$((32*3)) count=1 status=none
\end{lstlisting}
        \item[] Exercise: \texttt{dd | xxd | less}
\begin{lstlisting}[basicstyle=\tiny]
dd if=img_3.txt bs=512 skip=3 count=1 status=none | xxd | less
	0+1 records in
	0+1 records out
	55 bytes (55 B) copied, 5.04e-05 s, 1.1 MB/s

0000000: 4f76 6572 6865 6164 2031 3233 3435 3637  Overhead 1234567
0000010: 3839 3020 204d 6573 7361 6765 2d31 2020  890  Message-1  
0000020: 3039 3837 3635 3433 3231 2020 2020 2020  0987654321      
0000030: 2020 2020 2020 20
\end{lstlisting}
    \end{itemize}
\end{frame}


\begin{frame}[fragile]
  \frametitle{4.3 dd - disk imaging rudimentary}
    \begin{itemize}
        \item[] Demo: Continue an interrupted imaging process
\begin{lstlisting}[basicstyle=\tiny]
dd if=img_2.txt of=broken.raw bs=512 skip=0 count=2 status=none

ll  img_2.txt  ..... 1591 Aug 13 14:40 img_2.txt*
ll broken.raw  ..... 1024 Aug 13 15:05 broken.raw

dd if=img_2.txt of=broken.raw bs=512 skip=2 seek=2 status=none

md5sum  img_2.txt f319b1cc9d424a923a8c83c3e67185f1
md5sum broken.raw f319b1cc9d424a923a8c83c3e67185f1
\end{lstlisting}
        \item[] Error handling: Bad blocks
\begin{lstlisting}[basicstyle=\tiny]
$ dd if=img_3.txt of=out_3.txt bs=512 conv=noerror,sync
\end{lstlisting}
        \item[] Demo: Progress
        \begin{itemize}
            \item[] Option: \texttt{status=progress}
            \item[] Signaling: \texttt{'\&' and 'kill -10'}
        \end{itemize}
    \end{itemize}
\end{frame}


\begin{frame}[fragile]
  \frametitle{4.4 Disk acquisition}
    \begin{itemize}
        \item Forensic features
        \begin{itemize}
            \item Progress monitoring
            \item Error handling \& logging
            \item Meta data
            \item Splitting output files \& support of forensic formats
            \item Cryptographic hashing \& verification checking
        \end{itemize}
        \item Example: hashing
        \begin{itemize}
            \item[] \texttt{md5sum circl-dfir.dd} $\to$ \texttt{bd80672b9d1bef2f35b6e902f389e83}
            \item[] \texttt{sha1sum circl-dfir.dd} $\to$ \texttt{e5ffc7233a..........7e53b9f783}
        \end{itemize}
        \item Tools
        \begin{itemize}
            \item dd
            \item ddrescue, gddrescue, dd\_rescue
            \item dc3dd - Department of Defense Cyber Crime Center
            \item dcfldd - Defense Computer Forensic Labs
	    \item rdd-copy, netcat, socat, ssh
            \item Guymager
        \end{itemize}
    \end{itemize}
\end{frame}


\begin{frame}[fragile]
  \frametitle{4.5 Exercise: dc3dd}
  \begin{lstlisting}[basicstyle=\tiny]
dc3dd if=/mnt/ntfs/carving/deleted.dd                   # Input file
      log=usb.log  -/                                   # Logging
      hash=md5 hash=sha1  -/                            # Hashing
      ofsz=$((8*1024*1024)) ofs=usb.raw.000             # Chunk files of 8MB


ls -l


cat usb.log


cat usb.raw.00* | md5sum                                # Verify hashes
cat usb.raw.00* | sha1sum


dc3dd wipe=/dev/sdx                                     # Wipe a drive
  \end{lstlisting}
\end{frame}


\begin{frame}[fragile]
  \frametitle{4.6 SuashFS as forensic container}
    \begin{itemize}
        \item Embedded systems
        \item Read only file system
        \item Supports very large files
        \item Adding files possible
        \item Deleting, modifying files not possible
        \item Compressed
	\item[] $\to$ Real case: 3*1TB disks stored in 293GB container
	\item Bruce Nikkel: \url{http://digitalforensics.ch/sfsimage/}
    \end{itemize}
  \begin{lstlisting}[basicstyle=\tiny]
mksquashfs circl-dfir.dd case_123.sfs
mksquashfs analysis.txt case_123.sfs
unsquashfs -ll case_123.sfs
.....
mksquashfs analysis.txt case_123.sfs
.....
sudo mount case_123.sfs /mnt/
  \end{lstlisting}
\end{frame}


\begin{frame}[fragile]
  \frametitle{4.7 Exercise: Modify data on RO mounted device}
  \begin{lstlisting}[basicstyle=\tiny]
mount
mount -o ro,remount /media/michael/7515-6AA5/
mount


Demo: Modify Document

strings -td /dev/sdb1
    .....
    299106 Hello World!
    .....

echo $((299106/512))
    584

dd if=/dev/sdb1 bs=512 skip=584 count=1 of=584.raw
ll
hexer 584.raw

dd of=/dev/sdb1 bs=512 seek=584 count=1 if=584.raw
mount


Demo: Review Document
  
  
  \end{lstlisting}
\end{frame}


\begin{frame}
    \frametitle{4.7 Exercise: RO Countermeasures}
    \begin{itemize}
        \item Try on board methods:
        \begin{itemize}
            \item[]
            \begin{itemize}
                \item \texttt{hdparm -r1 /dev/sdb}
                \item \texttt{blockdev --setro /dev/sdb}
                \item \texttt{udev} rules
                \item[] $\to$ Attack on block device still possible
            \end{itemize}
        \end{itemize}
        \item Try Forensics Linux Distributions:
        \begin{itemize}
            \item Live Kali 2018\_4 in forensic mode
            \item SANS SIFT Workstation 3.0
            \item DEFT X 8.2 DFIR Toolkit
            \begin{itemize}
                \item Some distributions do not auto mount
                \item[] $\to$ Attack on block device still possible
            \end{itemize}
        \end{itemize}
\item Kernel Patch: Linux write blocker (not tested)
        \begin{itemize}
                \item[] $\to$ \url{https://github.com/msuhanov/Linux-write-blocker}
        \end{itemize}
        \item Hardware Write Blocker
        \begin{itemize}
            \item[]
            \begin{itemize}
                \item[] $\to$ Effectively block attack
            \end{itemize}
        \end{itemize}
    \end{itemize}
\end{frame}




