\begin{frame}
  \begin{center}
    {\bf Understanding RSA}
  \end{center}
\end{frame}

\begin{frame}
  \frametitle{RSA Basics}
  Ron {\bf R}ivest, Adi {\bf S}hamir, and Leonard {\bf A}dleman in 1977:
  \begin{itemize}
    \item asymmetric crypto system,
    \item can encrypt and sign,
    \item messages are big numbers,
    \item encryption is basically multiplication of big numbers,
    \item creates a \textit{trapdoor permutation}: turning x in y is easy, but
      finding x from y is hard.
  \end{itemize}

\end{frame}
\begin{frame}[fragile]
  \frametitle{RSA ``by hand''}
  \begin{itemize}
  \item {\bf Hands-on,} a sagemath script that is a toy example of RSA: 

\begin{lstlisting}
cd ~/hands-on/UsingRSA
sage rsa.sage
\end{lstlisting}
\item {\bf Outputs:}
\end{itemize}
\begin{lstlisting}[basicstyle=\tiny]
PlainText is: 1234567890
p = random_prime(2^32) = 2312340619
q = random_prime(2^32) = 2031410981
n = p*q = 4697314125248937239
phi = (p-1)*(q-1) = 4697314120905185640
e = random_prime(phi) = 2588085603940229747
d = xgcd(e,phi)[1] = -2102894211931680277
Does d*e == 1?
 mod(d*e, phi) = 1
CipherText y = power_mod(x, e, n) = 1454606910711062745
Decrypted CT is: 1234567890
\end{lstlisting}
\end{frame}

\begin{frame}[fragile]
  \frametitle{RSA - Use with openssl}
  \begin{itemize}
  \item {\bf Hands-on}:

\begin{lstlisting}
~/hands-on/UsingRSA
\end{lstlisting}

   \item Decrypt message.bin
   \item generate a new private key,
   \item generate the corresponding public key,
   \item use this new key to encrypt a message,
   \item use this new key to decrypt a message. 
   
  \end{itemize}
\end{frame}

\begin{frame}
  \frametitle{With only one key}
  Several potential weaknesses:
  \begin{itemize}
    \item Key size too small: keys up to 1024 bits are breakable given the
      right means,
    \item close p and q,
    \item unsafe primes, smooth primes,
    \item broken primes (FactorDB, Debian OpenSSL bug).
    \item signing with RSA-CRT (instead of RSA-PSS)
  \end{itemize}

\end{frame}

\begin{frame}
  \frametitle{With a set of  keys}
  Several potential weaknesses:
  \begin{itemize}
   \item share moduli: if n1 = n2 then the keys share p and q,
   \item share p or q,
  \end{itemize}
 \vspace{10mm}
  {\bf In both case, it is trivial to recover the private keys.}
\end{frame}


\begin{frame}[fragile]
  \frametitle{Breaking small keys\footnote{https://www.sjoerdlangkemper.nl/2019/06/19/attacking-rsa/}}
  \begin{itemize}
\item {\bf Hands-on}:

\begin{lstlisting}
~/hands-on/SmallKey
\end{lstlisting}

   \item what is the key size of smallkey?
   \item what is n?
   \item what is the public exponent?
   \item what is n in base10?
   \item what are p and q?
   
  \end{itemize}

  \vspace{8mm}
  {\bf Let's generate the private key: }using p, then using q.
  
\end{frame}

\begin{frame}[fragile]
  \frametitle{Close Prime Factors}
  \begin{itemize}
\item {\bf Hands-on}:

\begin{lstlisting}
~/hands-on/ClosePQ
\end{lstlisting}

   \item use Fermat Algorithm\footnote{\url{http://facthacks.cr.yp.to/fermat.html}} to find {\bf both p and q:}

\begin{lstlisting}[basicstyle=\tiny]
def fermatfactor(N):
  if N <= 0: return [N]
  if is_even(N): return [2,N/2]
  a = ceil(sqrt(N))
  while not is_square(a^2-N):
    a = a + 1
  b = sqrt(a^2-N)
  return [a - b,a + b]
\end{lstlisting}

  \end{itemize}
 
\end{frame}

\begin{frame}[fragile]
  \frametitle{Shared prime factors}
     Researchers have shown that several devices generated their keypairs
   at boot time without enough entropy\footnote{Bernstein, Heninger, and Lange: \url{http://facthacks.cr.yp.to/}}:
   
\begin{lstlisting}[language=python, basicstyle=\tiny]
prng.seed(seed)
p = prng.generate_random_prime()
// prng.add_entropy()
q = prng.generate_random_prime()
n = p*q
\end{lstlisting}

Given n=pq and n' = pq' it is trivial to recover the shared p by computing their
{\bf Greatest Common Divisor (GCD)}, and therefore {\bf both private
  keys}\footnote{\url{http://www.loyalty.org/~schoen/rsa/}}.\\
\vspace{5mm}
``They cracked about 13000 of them''
\end{frame}

\begin{frame}[fragile]
  \frametitle{Shared prime factors}
  \begin{itemize}
\item {\bf Hands-on}:

\begin{lstlisting}
~/hands-on/SharedPrimeFactor
\end{lstlisting}

\item Read README.txt, you have a challenge to solve :

  \begin{itemize}
  \item the \emph{answers} folder should be left alone for now,
  \item \emph{scripts} contains scripts that may be useful
    to solve the challenge,
  \item \emph{attempts} may hold your attempt are
    generating private keys. 
  \item \emph{bgcd-bd.sage} contains Daniel J. Berstein's algorithm for computing RSA
    collisions in batches.
  \end{itemize}

  \end{itemize}
 
\end{frame}
