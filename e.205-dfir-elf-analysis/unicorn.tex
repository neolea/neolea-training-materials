\begin{frame}
\frametitle{Decrypting files without access to a tool}
\begin{block}{Problem Statement}
    \begin{itemize}
        \item Faced with a large number of encrypted files.
        \item Encryption uses a custom implementation.
        \item No available command-line tool for decryption.
        \item The key and/or IV has been recovered.
        \item Debugging and manual decryption of each file is time-consuming and inefficient.
    \end{itemize}
\end{block}
\end{frame}

\begin{frame}
\frametitle{Decrypting Files Without Access to a Tool}
\begin{block}{Traditional Approach}
\begin{itemize}
    \item Write a loader program to execute the code.
    \item Read the code into a buffer.
    \item Cast the buffer to a function pointer.
    \item Execute the function pointer.
    \item \textbf{Challenges:}
    \begin{itemize}
        \item Buffers are often protected against code execution.
        \item Requires fiddling with \texttt{mmap} and \texttt{mprotect}.
        \item The code might include malicious instructions that went unidentified.
        \item The decryptor may be designed for another CPU architecture (e.g., MIPS, RISC-V).
    \end{itemize}
\end{itemize}
\end{block}
\end{frame}

\begin{frame}
    \frametitle{Decrypting Files Without Access to a Tool}
    \begin{itemize}
        \item The Unicorn Engine\footnote{\url{https://www.unicorn-engine.org/}} is a CPU emulator based on QEMU.
        \item Supports multiple architectures:
        \begin{itemize}
            \item ARM, ARM64 (ARMv8), m68k, MIPS, PowerPC, RISC-V, S390x (SystemZ), SPARC, TriCore, and x86 (including x86\_64).
        \end{itemize}
        \item Provides bindings for various programming languages:
        \begin{itemize}
            \item Pharo, Crystal, Clojure, Visual Basic, Perl, Rust, Haskell, Ruby, Python, Java, Go, D, Lua, JavaScript, .NET, Delphi/Pascal, and MSVC.
        \end{itemize}
        \item Offers hooking capabilities for:
        \begin{itemize}
            \item \textbf{Memory access}, \textbf{executed instructions}, and \textbf{interrupts}.
        \end{itemize}
        \item Thread-safe\footnote{Multithreading is often used as an anti-debugging technique.}
        \item Works without modifying code (e.g., no need to insert instructions such as \texttt{INT3} or \texttt{0xCC}).
    \end{itemize}
\end{frame}

\begin{frame}[fragile]
\frametitle{Building Unicorn Engine}
\begin{block}{Prerequisites}
    Install the required tools:
    \begin{itemize}
        \item \texttt{cmake}
        \item \texttt{pkg-config}
    \end{itemize}
    \textbf{Command:}
    \begin{verbatim}
sudo apt install cmake pkg-config
    \end{verbatim}
\end{block}

\end{frame}

\begin{frame}[fragile]
\frametitle{Building Unicorn Engine}

\begin{block}{Build Steps}
    Follow these steps to build Unicorn:
    \begin{enumerate}
        \item Create and navigate to the build directory:
        \begin{verbatim}
mkdir build; cd build
        \end{verbatim}
        \item Run \texttt{cmake} with the release build type:
        \begin{verbatim}
cmake .. -DCMAKE_BUILD\_TYPE=Release
        \end{verbatim}
        \item Compile the project and install it:
        \begin{verbatim}
make & make install
        \end{verbatim}

    \end{enumerate}
\end{block}
\end{frame}
